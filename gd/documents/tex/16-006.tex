\documentclass{gd-document}

\usepackage{url}

\begin{document}
\title{Input on Meeting Topics}
\author{skarfester}
\date{2016-03-23}
\docnr{16/6}
\abstract{\noindent
  [ A few comments by skarfester on several different points on the
    discussion schedule of the 2nd General Discussion.\\— note by
    Quintus ]}

\maketitle{}

\tableofcontents

\section{Audience + Target Age Group (Topic 6)}

The best option is to expand the target audience; there’s no need (as
a platform game) to focus in a particular age range. To achieve this
TSC needs different campaigns and/or difficulty levels.

\section{Depth of Story (Topic 7)}

A platformer doesn’t need much of a story, but is a good complement if
we integrate it in the game. For example, we can think the campaigns
as “missions”, like Indiana Jones films.

\section{Themes (Topic 8)}

This is related to theme and audience: adding more characters to
choose from. This way we can have a female character, and maybe more,
for example a “classic Alex”, if TSC changes Alex graphics.

\section{Level of Cuteness (Topic 9)}

Mantaining a relatively cute comic style works for me: I’d avoid too
childish graphics and there’s no need for something “agressive”,
“dark” or “hipster”. Cute works for most people.

\section{Narration, cinematic scenes, and story formatted text (Topic 10)}

If we develop a story we need to introduce it in the game. I’ll stick
only to visual narrative, without text or speech. That way we can
mantain TSC more language independent, and avoids lot of translation
work. The problem with this is that a complex story can be hard to
explain without words.

\section{Life energy system (Topic 11)}

That’s the big issue. I’ll recap some opinions discussed in
\url{http://forum.secretchronicles.de/topics/367} and others:

\begin{itemize}
\item quit lives: the progress is saved so they have no sense.
\item save points, save states, checkpoints: I’ll suggest to
  automatically save after a completed level. This way the player
  doesn’t need to care about that and can continue playing or quitting
  the game without losing (worldmap) progress. In addition to that I
  suggest a new power-up wich acts as a checkpoint. With that level
  developers can design larger levels without the risk of becoming too
  frustating. We can add as well arbitrary save-states bought by the
  players with diamonds, for special ocasions.
\item diamonds as a currency: if we cannot earn lives, we can use
  diamonds for buying regular or special items, like save states or
  “flying” ability to pass difficult areas (very expensive). We’ll
  need an intem menu for this (the pause menu is enough).
\item quit score points?: they’ll have no impact in the game, so we
  can keep as hi-score ranking information or remove completely.
\item quit game over screen: obvious. When die, you just go back to
  the worldmap or the previous checkpoint or save state.
\item energy system: I prefer a “hit system”, because is more simple
  and straightforward than a energy bar with different damage
  levels. Now we have small Alex, big Alex and Alex with powers (wich
  doesn’t count as an extra hit). In my opinion is a good idea to
  reflect the hit status in Alex itself (like small/big Alex) for
  example in his shoes: red shoes=1 hit left, yellow shoes=2 hits,
  green=3… We can count the extra powers (fire, ice) as another hit,
  just losing them when hit.
\item difficulty levels: with a hit system is easy to implement a
  difficulty level: starting with 3 hits for kids, 2 for average
  players and 1 for pros but… I’m not a fan of difficulty levels. Once
  you end a game in “normal” you probably won’t play it again in
  “difficult”, because the levels are going to be essencially the
  same. I prefer games with difficulty depending on campaigns or
  levels. Maybe we can rely on level design and add just a “kid”
  option with extra hits.
\item pits (no floor): I think we should let empty pits kill Alex
  instantly, but using them only on special ocasions, such as
  castle/boss levels. This adds extra challenge to certain moments
  because even if you are full of energy, a mistake can kill you. We
  should use pits with static enemies instead, with a way to jump
  back. I know I rely a lot on level design…

\section{Closing}

hank you for reading. These are just my views, and I’m not close to
other options. I hope I’ve given you a couple ideas for the game.

\end{itemize}

\end{document}

%%% Local Variables:
%%% mode: latex
%%% TeX-master: t
%%% End:
