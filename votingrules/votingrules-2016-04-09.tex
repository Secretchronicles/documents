% Secretchronicles Voting Rules Draft
% Copyright © 2015 The TSC developers
% All Rights Reserved.
%
% This document is intended to be processed by LuaLaTeX, any other TeX
% engines most likely won’t work.
%
%   $ lualatex votingrules.tex
%   (multiple times if needed)
%
% Note you need to have the Libertine font family (Linux Libertine
% and Linux Biolinum) installed; modern Linux distros should provide
% a “otf-libertine” or “ttf-libertine” (OTF preferred) package. See
% the \setmainfont and \setsansfont commands below for changing the
% font if you need to.
%
% The default paper size is ISO A4. Change “a4paper” in the
% documentclass options to “letter” or whatever your printer can do
% if this is not the correct format for you.
%
% Changelog:
% - 2015-10-08: First accepted version of this document.

\documentclass[10pt,a4paper,DIV=calc,headings=medium,twocolumn,final]{scrartcl}
\usepackage[english]{babel}
\usepackage{fontspec}
\usepackage{scrjura}
\usepackage{xspace}
\usepackage{draftwatermark}

%\SetWatermarkText{DRAFT}

\newcommand\irc{\textsc{irc}\xspace}
\newcommand\eg{e.\,g.\xspace}
\newcommand\ie{i.\,e.\xspace}

\renewcommand\thesection{\Roman{section}}
\addtokomafont{disposition}{\centering}

\setmainfont{Linux Libertine O}
\setsansfont{Linux Biolinum O}

\KOMAoptions{DIV=last}

\begin{document}
\title{Secretchronicles Project Voting Rules}
\subtitle{Version from 2015-08-15, last changed with vote results
  of the 3\textsuperscript{rd} vote from 2016-04-09
  (Decl. of Res. from 2016-04-09), in force starting 2016-04-09.}
\date{August 15, 2015}
\maketitle{}

\section*{Preamble}

This document outlines the modus, rules, and majority requirements for
any votes taken in the course of the development of the
Secretchronicles project. It is intended to minimise possible
conflicts due to disagreements on the voting procedure by being a
definitive description approved by the TSC development team as a
whole.

\section{The Team}

\begin{contract}
  \Clause{title=Team member}

  Team member is for the sake of this document who is listed
  publically as a member of the “Secretchronicles” GitHub
  organisation.

  \Clause{title=Acquisition of Membership}

  The team decides by vote whether any person should be granted team
  membership by adding him to the GitHub organisation. Before this
  vote can be called for, the person in question has to either

  \begin{enumerate}
  \item have five commits included into the main development branch of the
    game, or
  \item have been nominated by an existing team member.
  \end{enumerate}

  The vote requires that two thirds of the team members voting are in
  favour of adding the new person to the team.

  In case of Nr. 1 above, as an exception to \ref{cl:voting}, the
  person with this amount of commits included is also allowed to place
  the vote request on the project lead.

  \Clause{title=End of Membership}

  There are three ways by which a person can be removed from the team:

  \begin{enumerate}
  \item The person can request removal formally by approaching the
    project lead accordingly, or
  \item if the person does not show any activity for a year, he will
    be removed.  The minimal activity required is one forum/tracker
    post (advancing TSC discussions) or commit in source control per
    year. When two weeks are remaining before a person reaches the one
    year mark for inactivity, the project lead will send an email to
    the inactive person asking about their involvement.  If they do
    not make a posting or commit, or if they do not respond, they will
    be removed from the team.
  \item The team reserves the right to remove any person at any time,
    for any reason, through a vote.  This requires a two thirds
    majority of those voting, and the candidate for removal is not
    allowed to participate in the vote.  The team is expected to try to
    follow fair criteria for such removals and consider changing the
    rules if repeated removals occur.
  \end{enumerate}

  Upon removal a person will be removed from the GitHub organization.
  They can be readded at any time by meeting the criteria for a
  regular team addition.

  \Clause{title=General Rule of Consent}

  All decisions are to be made in consent of all team members. Only if
  it is impossible to reach a consent even by means of compromises, a
  voting shall be held.
\end{contract}

\section{Voting process}

\begin{contract}

  \Clause{title=Announcement of Vote}\label{cl:voting}

  Any team member (proponent) can request the project lead to organise
  a vote on a subject the member deems impossible to decide on in
  consent. The project lead then announces the upcoming vote
  sufficiently early so that every team member has time to prepare, at
  least two weeks in advance (Announcement of Vote).

  The announcement must include:

  \begin{enumerate}
  \item The proponent,
  \item The vote subject,
  \item The vote options, and
  \item The beginning and ending dates of the voting period.
  \end{enumerate}

  If the project lead is unavailable, the assistant lead guides the
  process instead.

  \SubClause{title=IRC Voting Announcement}

  For votings held in \irc — as an exception to §\,5(2)(Nr.3) — it is
  sufficient to announce all available voting options in \irc prior to
  taking the votes.

  If at least three team members are available in \irc and are not just
  idling, it is possible to waive the two-week preparation interval
  and vote immediately. The waiver requires all non-idling team
  members in the chatroom to agree on it.


  Any team member can challenge a voting made under the conditions of
  (2) above until two weeks after the declaration of results have
  passed by approaching the project lead accordingly. If the voting is
  challenged, it is treated the same way as if it failed (\ref{failedvote}).

  \Clause{title=Open Vote}

  The default open vote happens openly on any of the project’s
  communication channels, \eg in \irc or on the tracker. Each team
  member votes by stating the option he wants to vote for.

  \Clause{title=Covert Vote}

  After the Announcement of Vote has been published, any one team
  member can request a covert vote. The request must be honoured by
  public announcement.

  In the covert vote each member votes by means of a technical
  procedure that ensures nobody can find out later who voted on which
  option.

  \Clause{title=Technical details}

  The technical details of the open and covert vote are worked out by
  the project lead and announced prior to each vote.

  \Clause{title=Call for Votes}

  When the voting period has started, the project lead publishes this
  fact in all communication channels where the original Announcement
  of Vote was posted (Call for Votes).

  The Call for Votes must repeat the termination date of the voting
  period.

  Any votes made before the Call for Votes has been published are
  invalid.

  \Clause{title=Results}

  Unless otherwise demanded by this document, the option that receives
  the simple majority (\ie most votes) is the winner option. It is a
  binding decision that determines how to proceed in the given topic
  and prevents any further vote on the topic given all circumstances
  equal.

  \Clause{title=Second Vote}

  If no single option receives the most votes, a second vote is held
  with the set of options reduced to those options that received the
  most votes. The preceeding clauses are in force accordingly.

  \Clause{title=Failed vote}\label{failedvote}

  A vote is failed if nobody participates, no single option receives
  the most votes in the second vote, the proponent revokes the
  voting suggestion, or the winning option does not receive the
  majority required.

  A failed vote has no binding effects; the same subject might be
  voted over again.

  \Clause{title={Declaration of Results}}

  After the vote is over, the project lead announces the result, if
  any.
\end{contract}

\section{Complaints}

\begin{contract}
  \Clause{title=Right to Complain}

  Until 14 days since the declaration of results have passed, each
  team member has the right to request a check as to whether the
  voting procedure outined by this document was followed properly.

  \Clause{title=Procedure}

  The team member directs his complaint to the project lead or to the
  project assistant lead (examiner) and has to state why he thinks
  this document has been violated. The project lead and the assistant
  lead may not decide about their own complaints; if they’re both
  involved, a person is randomly chosen from a set of noninvolved
  volunteers.

  The examiner has to forward it to the team member being complained
  about without further comment, and has to ask him for a statement on
  the issue within a reasonable timeframe. This statement, and any
  further replies, must be made available to all parties involved.

  If the examiner deems it necessary, he can request a live chat in
  \irc in which all parties participate prior to a decision.

  \Clause{title=Decision}

  The examiner has to decide within reasonable time, but at most
  within two months after the complaint was filed.

  The decision needs to contain the reasons for why or why not the
  complaint is valid.

  \Clause{title=Effect}

  A successful complaint results in the vote being failed as per
  \ref{failedvote}. The examiner should state this effect in his
  decision.

  \Clause{title=Ne bis in idem}

  The same vote may only be complained about once. If multiple persons
  complain about the same vote, the examiner has to merge all
  complaints into one and decide on all of them together.
\end{contract}

\section{Final clauses}

\begin{contract}
  \Clause{title=Archive}

  All votes, their results, and their associated complaints and their
  results, are for reference archived publically by the project lead.

  \Clause{title=Changes}

  Changes to this document require a vote whose winning option passes
  with at least two thirds of all existing team members being in
  favour of it.

  \Clause{title=First Date of Force}

  This document is in force from the next day after it was accepted by
  at least two thirds of the current team members\footnote{This
    document was accepted by the team with vote results from 2015-10-08.}.
\end{contract}

\end{document}


%%% Local Variables:
%%% mode: latex
%%% TeX-master: t
%%% End:
